\documentclass[11pt,letterpaper]{article}
\usepackage[utf8]{inputenc}
\usepackage[left=1in,right=1in,top=1in,bottom=1in]{geometry}
\usepackage{amsfonts,amsmath}
\usepackage{graphicx,float}
\input{~/code/math_commands.tex}
% -----------------------------------
\usepackage{hyperref}
\hypersetup{%
  colorlinks=true,
  linkcolor=blue,
  citecolor=blue,
  urlcolor=blue,
  linkbordercolor={0 0 1}
}

% -----------------------------------
\usepackage{titlesec}
\renewcommand\thesubsection{(\arabic{section}.\alph{subsection})}
\titleformat{\subsection}[runin]
        {\normalfont\bfseries}
        {\thesubsection}% the label and number
        {0.5em}% space between label/number and subsection title
        {}% formatting commands applied just to subsection title
        []% punctuation or other commands following subsection title
% -----------------------------------
\setlength{\parindent}{0.0in}
\setlength{\parskip}{0.1in}
% -----------------------------------
\begin{document}

We review methods for finding roots of nonlinear functions such as the fixed-point method and Newton's method. We will do a mix of coding and theory exercises. 

\section{Fixed point methods}
We want to find the roots of the nonlinear equation
\begin{align*}
    f(x) := x^2-x-2 = 0
\end{align*}
using the fixed point method.

\vspace{1cm}
We will try to find the positive root ($x^*$) through the fixed point iteration of the form $x_{k+1} = g(x_k)$. We investigate two choices:
\begin{itemize}
    \item $g_1(x) = x^2-2$
    \item $g_2(x) = \sqrt{x+2}$
\end{itemize}

\subsection{}\label{sec:1c}
Verify that $x^*$ is indeed fixed points for the two functions. That is, $x^* = g(x^*)$. 
\begin{enumerate}
    \item You could verify this with a plot. (If there are two lines in a plot, put legends on them.)
\end{enumerate}

\subsection{}
Will both choices work in the fixed point algorithm to find the root $x^*$? (Hint: consider the stability of the fixed point.)
\begin{enumerate}
    \item Could you determine the stability using the plot you made in 1.a?
\end{enumerate}

\subsection{}
Implement the fixed point method for both choices of $g(x)$ above. 
\begin{enumerate}
    \item Set $x_0 = 5$.
    \item What would be some good termination criteria for the fixed point algorithm?
    \item Try to ``generalize'' your code. For example, what if we want to pick a new function $g(x)$?
\end{enumerate}

Now consider the convergence behavior of the stable fixed point algorithm. 

\subsection{}
From theory alone, how fast do you expect the convergence? (linear/superlinear/quadratic)? What is the expected convergence rate?

\subsection{}
Is this what your numerical results show? How do you verify the convergence behavior numerically? Try to show this with a plot. 
\begin{enumerate}
    \item Plot a representation of the error $|x_k-x^*|$ against iteration numbers. Which methods of plotting should we use? \texttt{plot}, \texttt{semilogy}, or \texttt{loglog}?
    \item In more complicated examples we do not know $x^*$. What should we plot on the $y$-axis? Can you relate these quantities to the actual error $|x_k-x^*|$?
    \item Did you remember to add a title and axis label to your plot?
\end{enumerate}

\subsection{}
Could you diagnose the convergence rate numerically? Does it match the theoretical expectation?

\subsection{}
Is it possible to use $g_2(x)$ in the fixed-point iteration to find the negative root? Could we modify it so that it would work? In your spare time, you could verify the convergence behavior to the negative root as an exercise. 

\newpage
\section{Newton's method and roots with higher multiplicity}
The convergence theorem for Newton's method requires the first derivatives at the root $x^*$ to be nonzero. We will explore what happens when $f(x^*) = 0$.

\subsection{}
Take $f(x) = x^2$. Show that Newton’s method applied to this function gives linear convergence to the root of $f$ (hint: just plug $f$ into the formula for the Newton iterates and do some algebra). 

To further study this, we first define multiplicity: $x^*\in \mathbb{R}$ is a root of multiplicity $m$ for the equation $f(x)=0$ if there is a function $h(x)$ such that $h(x^*)\neq 0$ and $f(x) = (x-x^*)^m h(x)$.

\subsection{}
Suppose that a function $f$ has $m$ continuous derivative on the interval $(a,b)$ containing $c$ (i.e. $f(x)\in C^m_{(a,b)}$). Show that $f$ has a zero of multiplicity $m$ at $c$ if and only if
\begin{align*}
    0 = f(x^*) = f'(x^*) = \dots = f^{(m-1)}(x^*)
\end{align*}
and
\begin{align*}
    f^{m}(x^*) \neq 0.
\end{align*}

\subsection{}
Suppose $x^*$ is a zero of multiplicity $m$ of $f$, and $f(x)\in C^m_{(a,b)}$, $x^*\in (a,b)$. Show that the following fixed-point method has $g'(x^*) = 0$:
\begin{align*}
    g(x) = x-m\frac{f(x)}{f'(x)}.
\end{align*}
What can you say about the convergence behavior of this fixed-point method?

\subsection{}
Code this modified Newton's method to solve $f(x) = x^2 = 0$. Check the convergence numerically. How do you plan to show this?
    
    
% \bibliographystyle{alpha}
% \bibliography{citation}


\end{document}