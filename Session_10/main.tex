\documentclass{article}%ctex
\input{~/code/math_commands.tex}




\title{\huge Session 10\\
\normalsize}
\author{Xuanxi Zhang}
\begin{document}
\maketitle

\section{Vandermonde matrix}
The interpolation problem is, given a smooth function $f$, to find an order $n$ polynomial $p_n(x) = \sum_{j=0}^n a_j x^j$ equal to $f$ on $\{x_0, x_1, \cdots, x_n\}$, which can be converted to a linear system:
$$
\begin{bmatrix}
1 & x_0 & x_0^2 & \cdots & x_0^n\\
1 & x_1 & x_1^2 & \cdots & x_1^n\\
\vdots & \vdots & \vdots & \ddots & \vdots\\
1 & x_n & x_n^2 & \cdots & x_n^n
\end{bmatrix}
\begin{bmatrix}
a_0\\
a_1\\
\vdots\\
a_n
\end{bmatrix}=\begin{bmatrix}
f(x_0)\\
f(x_1)\\
\vdots\\
f(x_n)
\end{bmatrix}
$$
Here the matrix on the left is called the Vandermonde matrix, which is also very important in quantum mechanics. We will proof the existence and uniqueness of $p_n(x)$ by studying the Vandermonde matrix.
\subsection{}
$$
\det\left(\begin{bmatrix}
  1 & x_0 & x_0^2 & \cdots & x_0^n\\
1 & x_1 & x_1^2 & \cdots & x_1^n\\
\vdots & \vdots & \vdots & \ddots & \vdots\\
1 & x_n & x_n^2 & \cdots & x_n^n
\end{bmatrix}\right)=\det\left(\begin{bmatrix}
  1 & x_0 & x_0^2 & \cdots & x_0^n\\
1 & x_1 & x_1^2 & \cdots & x_1^n\\
\vdots & \vdots & \vdots & \ddots & \vdots\\
1 & x_n & x_n^2 & \cdots & x_n^n
\end{bmatrix}\begin{bmatrix}
  1&-x_0&&&\\ 
  &1&-x_0&&\\
  &&\ddots&\ddots&\\
  &&&1&-x_0
\end{bmatrix}\right)
$$
What is the result of matrix product on the right?
\subsection{}
Prove that for a Vandermonde matrix $V$, $\det(V)=\prod_{0\leq i<j\leq n}(x_j-x_i)$, by conduction.
\subsection{}
Conclude that when $x_i$ are distinct, the Vandermonde matrix is invertible.

\section{Interpolation basics}
\begin{itemize}
\item True or False? For the nodes $x_0=0, x_1=1, x_2= 2$, the
  Lagrange interpolation polynomial $L_0(x)$ is $-x^2 + 1$.
\item True or False? We compute the Hermite interpolant with 3 distinct nodes of
  a function $f$ that is a polynomial of degree $4$. Then this
  Hermite interpolant is identical to $f$. (In short: Hermite
  interpolation with 3 nodes is exact for polynomials of degree 4.)
\item True or False? Hermite interpolation with 4 distinct nodes is exact for
  polynomials of degree 6.
\item True or false: Let $p_n$ be the Lagrange interpolant to a
function $f$ with $n+1$ interpolation points, and $e_n(x) =
|p_n(x) - f(x)|$.  The interpolation error $\|e_n\|_\infty$ {\em
always} gets arbitrarily small for large $n$, i.e.,
$\|e_n\|_\infty \rightarrow 0$ as $n \rightarrow \infty$.
\end{itemize}

\section{Lagrange interpolation polynomial example}
Let $x_0,\ldots, x_n$ be distinct interpolation nodes, and let
  $$
  p_n(x) = \sum^n_{k=0}L_{k}(x)(x_k)^j,
  $$
where $j$ is an integer and $n \geq j>0$. What is the $p_n(x)$ function? What are the values of $p_n(0)$ and $p_n(1)$?

\section{Hermite interpolation polynomial example}
Recall that the Hermite interpolation of a function $f$ at the points $x_0,x_1,x_2$ has the form 
$$p(x) = \sum_{j=0}^2H_j(x)f(x_j)
+ \sum_{j=0}^2K_j(x)f'(x_j).$$ 
where 
$$
H_j(x) = (1-2(x-x_j)L_j'(x_j))L_j^2(x),\quad K_j(x) = (x-x_j)L_j^2(x).
$$
\subsection{}
Show that the polynomial
$$ -\frac{1}{\pi}x^2 + x$$ 
is the Hermite interpolation polynomial of $f(x):=\sin(x)$ based on the nodes $x_0=0$, $x_1=\pi$.
  
\subsection{}
Calculate all 4 Hermite basis for the nodes $x_0=0$, $x_1=1$. This might be useful for the next section.


\section{spline interpolation}
Regular interpolation methods suffer from several issues:
\begin{itemize}
  \item They cannot guarantee that the interpolation error will go to zero for all functions as we use more interpolation points. For example, the Runge phenomenon.
  \item Higher degree polynomials require more computations.
\end{itemize}

To address these issues, we often use piecewise polynomial interpolation, known as spline interpolation. There are two common types of spline interpolation: Lagrange and Hermite splines. The former is the simplest $C$ interpolant, while the latter is the simplest $C^1$ interpolant.







\end{document}