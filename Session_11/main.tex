\documentclass{article}%ctex
\input{~/code/math_commands.tex}




\title{\huge Session 11\\
\normalsize}
\author{Xuanxi Zhang}
\begin{document}
\maketitle

\section{Deriving a new quadrature rule}
Given $f:[0,1]\rightarrow \mathbb{R}$, you want to derive a new
  quadrature rule that does uses not only function values, but also
  gradient values:
\begin{equation}
    \int_0^1 f(x)\,d x \approx \alpha_0 f(0) + \alpha_1 f'(0) + \alpha_2 f(1). \label{eq:11_3a}
\end{equation}
  
\subsection{}
First, find polynomials $J_0, J_1, J_2 \in \mathcal{P}_2$, with the
  following properties:
  \begin{align*}
    & J_0(0)=1,\quad J_0'(0)=0,\quad J_0(1)=0\\
    & J_1(0)=0,\quad J_1'(0)=1,\quad J_1(1)=0\\
    & J_2(0)=0,\quad J_2'(0)=0,\quad J_2(1)=1.
  \end{align*}
(Hint: For each $J_i$, make an ansatz for a quadratic polynomial using the monomial basis.)
Given $f$, you can now define a polynomial approximation $p\in \mathcal{P}_2$ via 
\begin{align}
    p(x)=f(0)J_0(x)+f'(0)J_1(x)+f(1)J_2(x). \label{eq:11_3b}
\end{align}
The polynomial $p$ is an approximation to $f$ in the sense that
$p(0)=f(0)$, $p'(0)=f'(0)$ and $p(1)=f(1)$. 

\subsection{}
Use the polynomial
  $p$ derived above and the same method used to derive the Newton-Cotess
  quadrature rules, to find the coefficients $\alpha_0$, $\alpha_1$ and
  $\alpha_2$ in \eqref{eq:11_3a}.
  
\subsection{}
Use your new quadrature rule to approximate
  $\int_0^1\exp(2x)\sin^2(x)\,dx$, and also compare with
  Simpson's rule. The exact value of this integral is 1.2668\dots. Note $e^2\sin^2(1)\approx 5.232$, $e^1\sin^2(\half)\approx 0.6248$


\section{Trapezoidal rule for smooth periodic functions}

We investigate how the (composite) trapezoidal rule performs for smooth, periodic functions. Consider integrating the smooth, periodic function $f(x) = e^{\sin x}$ over a single period. The exact value of the integral is
\[
I(f) = \int_0^{2\pi} e^{\sin x} \, d x = 7.95492652101284527\dots.
\]

\subsection{}
Write down the composite trapezoidal rule $T_N(f)$ on equispaced nodes $0=x_0 \le  \dots \le x_N = 2\pi$ for estimating the value of this integral. 

\subsection{}
Simplify your expression for $T_N(f)$ using the periodicity of $f$.
Show that $T_N(f)$ is equivalent to both a left-endpoint Riemann sum and a right-endpoint Riemann sum approximation to $I(f)$.



\section{Convergence order of quadrature}
\subsection{}
We would like to integrate a function on $[0,1]$ using the composite trapezoid rule with sub-interval size $h$. We name the result $T_h$. How does the error ($e_h = |T_h-I|$) scale with $h$?

\subsection{}
let $f(x)=e^{\sin(x)+\pi \cos(x)}+e^x$
Compute $T_h(f)$ for various progressively larger $N$. Plot the quadrature errors against $h$ on a log-log plot.
\subsection{}
In real application, we do not know the true value of the integral. To verify the order of convergence of our code, we can calculate this quantity:
\begin{align*}
    \frac{T_h-T_{h/2}}{T_{h/2}-T_{h/4}}.
\end{align*}
How does this quantity scale with $h$?

\subsection{}
show this by code using the same $f$.




\newpage
\section*{Solutions}
\subsection*{1.1}
The polynomials $J_0, J_1, J_2 \in \mathcal{P}_2$ with the given properties can be found as follows:
\begin{align*}
    J_0(x) &= 1 - x^2  \\
    J_1(x) &= x - x^2  \\
    J_2(x) &= x^2 .
\end{align*}
2/3, 1/6, 1/3

1.744, 1.2885333333


\end{document}