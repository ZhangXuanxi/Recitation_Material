\documentclass{article}%ctex
\input{~/code/math_commands.tex}




\title{\huge Session 8\\
\normalsize}
\author{Xuanxi Zhang}
\begin{document}
\maketitle



\section{Gershgorin disks and the power method}

Consider the matrix 
$$
A = \begin{bmatrix}
- 6 & 2 & 0.3 & 0 & -0.7 \\
2 & - 4 & 0.1 & 0.05 & 0 \\
0.3 & 0.1 & 2 & 0.1 & 0.1 \\
0 & 0.05 & 0.1 &  4 & 0 \\
-0.7 & 0 & 0.1  & 0  & 6
\end{bmatrix}
$$
and recall the definition of the Gershgorin disks:
\[
D_i = \{ z \in \mathbb C ~|~ |z - a_{ii}| \le \sum_{j \ne i} |a_{ij}| \}.
\]

Review of Complex Numbers: 
$z\in\sC$ can be represented as $z=x+iy$, where $x,y\in\sR$ is called real part and image part, $i$ is the imaginary unit, $i*i=-1$. The modulus of $z$ is defined as $|z|=\sqrt{x^2+y^2}$.

\subsection{}
Argue that all eigenvalues of $A$ are real.

\subsection{}
What are the Gershgorin disks for $A$?  Give a set, $D \subset \mathbb{R}$, that contains all eigenvalues of $A$.

\subsection{}
Can you conclude that the eigenvalue with the largest absolute value is simple?

\subsection{}
Argue that $A$ is invertible. Conclude that all diagonally dominant matrix is invertible.

\subsection{}
True or False? Let $A \in \mathbb{R}^{n \times n}$ and $D_i$, $i
  = 1,2,\dots,n$, be the Gerschgorin disks of $A$. If $0 \in
  \bigcup_{i=1}^n D_i$ then $A$ is singular.
  
\subsection{}
Write down the first iteration of the power method starting from $\vx_0 = (0,0,0,0,1)^T$.
You don't need to normalize.
Explain why $\vx_0 = \vzero$ is not a suitable starting point.

\subsection{}
The eigenvalues of $A$, after rounding, are $\{-7, -3, 2, 4, 6\}$.
Which eigenvalue direction will the sequence of the previous question converge to?

\

\section{Computing eigenvalues via the Power Iteration}
Given the following matrix:
\[
A=\begin{bmatrix}
-2 & 1 & 4  \\ 1 & 1 & 1 \\ 4 & 1 & -2 \end{bmatrix},
\]
It has eigenvalues and eigenvectors:
\[
\lambda_1=0,
\, {\boldsymbol v_1}
=\begin{bmatrix}
0.41  \\ -0.82 \\ 0.41 \end{bmatrix},\quad
\lambda_2=-6,
\,{\boldsymbol v_2}
=\begin{bmatrix}
0.71  \\ 0.0 \\ -0.71 \end{bmatrix},\quad
\lambda_3=3, \,{\boldsymbol v_3}
=\begin{bmatrix}
-0.58  \\ -0.58 \\ -0.58 \end{bmatrix}.
\]

\subsection{}\label{sec:1.a}
Calculate the first iterate of the power method when ${\boldsymbol x_0}=(0,1,1)^T$.

\subsection{}
Which eigenvalue direction will the sequence start from above $x_0$ converge to?
  
\subsection{}
Give an initialization vector such that the power method does \emph{not} converge to the direction of the largest (in absolute value) eigenvalue.
  
\subsection{}\label{sec:1d}
Write a simple program implementing the power method for the matrix $A$.


\section{The Inverse Iteration}
Take $A$ to be the matrix above, and let $\theta\in\mathbb{R}$ and let ${\boldsymbol x_0}\in\mathbb{R}^3$.

\subsection{}
If $\theta=2$, where will the sequence defined in (i)
  converge to and why?

\subsection{}
If $\theta=-2$, where will the sequence defined in (i)
  converge to and why?

\subsection{}
Write a simple program implementing the inverse power method.

\section{}
What is the flops for Computing

\begin{itemize}
  \item $(I-2vv^T)x$
  \item $x-(2vv^T)x$
  \item $x-2*v(v^Tx)$
\end{itemize}




\end{document}
